\chapter*{Estratto in lingua Italiana}
\label{cha:estratto}
\markboth{Estratto}{}
\addcontentsline{toc}{chapter}{Estratto in lingua Italiana}

L'imaging a risonanza magnetica (\ac{MRI}) è oggi una delle più comuni tecniche di generazione di immagini mediche, per via della natura non invasiva di questo tipo di scansione che permette di acquisire varie modalità (o sequenze) della parte del corpo scansionata, ognuna diversa dall'altra per livello di risoluzione e contrasto utilizzato.  Tuttavia, non è sempre possibile ottenere tutte le sequenze richieste, a causa di molteplici problemi, tra cui i tempi di scansione proibitivi o le allergie dei pazienti agli agenti di contrasto. Per ovviare a questo problema e grazie ai recenti miglioramenti nel campo del Deep Learning, negli ultimi anni i ricercatori hanno studiato l'applicabilità delle Reti Antagoniste Generative (\ac{GAN}) alla generazione delle modalità mancanti. Il nostro lavoro propone uno studio dettagliato che mira a dimostrare l'abilità delle GAN nel generare scansioni MRI realistiche di tumori cerebrali attraverso l'implementazione di diversi modelli. Abbiamo allenato in particolare due tipi di reti che differiscono per il numero di sequenze ricevute in input, utilizzando un dataset composto da 274 diversi volumi appartenenti a pazienti con tumori cerebrali e, tra una serie di diverse metriche implementate per valutare i nostri risultati, abbiamo validato la qualità dell'immagine generata dalla rete usando anche un modello di segmentazione.
Inoltre, abbiamo analizzato le \ac{GAN} addestrate, eseguendo alcuni esperimenti per capire come il contenuto informativo ricevuto in input passi attraverso il generatore, ovvero una delle due reti neurali che compongono una GAN. I nostri risultati dimostrano che le sequenze sintetizzate sono altamente accurate, realistiche e in alcuni casi indistinguibili dalle immagini provenienti dal dataset, evidenziando il vantaggio dei modelli multi-input che, rispetto a quelli single-input, possono sfruttare la correlazione presente tra le sequenze che sono disponibili. Inoltre, dimostrano l'efficacia delle skip connections e il loro ruolo fondamentale nel processo generativo mostrando come, spegnendo o perturbando i canali, le prestazioni della rete subiscano un calo significativo.


